\documentclass[12pt]{ctexart}
\def\version{1.0}
\newenvironment{smaller}{\zihao{5}}{\zihao{-4}}
% set the left/right margin such that the main title can be written within one line
\usepackage[left=30mm]{geometry}
\usepackage{enumitem}
\AddEnumerateCounter{\chinese}{\chinese}{}
\usepackage{fancyhdr}
\usepackage{graphicx}
\usepackage{longtable}
\fancypagestyle{runningpage}
{
  \fancyhead{}
  \fancyhead[C]{格致计划2023冬季读书营}
  \fancyfoot{}
  \fancyfoot[C]{第 \thepage 页}
}

\def\CurriculumScheduleWidth{1.6cm}
\title{神奇的熵}
\author{领读人:赵丰}
\date\today
\pagestyle{runningpage}
\begin{document}
% first page is cover

\maketitle
%\thispagestyle{empty}



\section{课程简介}
当我们描述一个不可逆的过程时,比如越来越乱的屋子、世上没有卖后悔药等等,往往会使用熵增这个概念,
这是我们在谈论物理学上的熵。你有没有发现,一张背景很黑的照片,相比一张阳光下的景物照,
文件大小更小,我们会用信息量的大小来解释,这是我们在谈论物理学里的熵。

熵为什么会自发地增加?数据最多可压缩到什么程度?玻尔兹曼提出的物理熵和香农提出的信息熵有什么联系?你了解熵背后的科学故事吗?
如果你对这些问题感兴趣,那么这门半读书半讲授的课程
适合你。

本门课程属于科学与技术的主题范畴,主要介绍物理熵和信息熵,它们的由来、概念本身与广泛的应用。
其中物理熵的部分将主要参考《溯源探幽:熵的世界》前三章进行讲解,
而信息熵的部分将主要参考《信息简史》中引子、第6、7章。


\section{课程目标}
物理熵在物理、化学等领域有广泛的应用,
而信息熵主要在通信与计算机科学中使用。
物理熵和信息熵不是两个完全不同的概念,
它们可以从数学模型和历史发展(科学史)的角度建立联系。
我希望通过关于这个主题的读书讨论,使大家能够了解物理熵和信息熵,以及
它们的联系。

\section{参考资料}
\begin{enumerate}
\item 冯端, 冯少彤. 溯源探幽:熵的世界. 科学出版社, 2005 (课上阅读材料)
\item (美)詹姆斯·格雷克. 信息简史. 人民邮电出版社, 2013 (课上阅读材料)
\item James Gleick. The Information: A History, a Theory, a Flood. Knopf Doubleday Publishing Group, 2011
\end{enumerate}

\section{课程安排}
读书营共3天的时间,计划第一天和第二天前半段时间带领学员阅读《溯源探幽:熵的世界》前三章,
第二天后半段时间和第二天带领学员阅读《信息简史》中引子、第6、7章。
主要采用带着问题阅读的方法,设置一些非开放的问题让学员在阅读中找答案。

文本细读过程中设置的讨论问题举例如下:


\subsection{溯源探幽:熵的世界}

阅读1-3页,试解释说明瓦特发明蒸汽机属于哪种范式,科学驱动技术还是技术驱动科学?

上述问题亦可从下面一段阅读材料中找到答案(技术驱动科学):

``James Watt’s 1769 invention of his steam engine to solve the problem of pumping water out of British coal mines.
These familiar examples deceive us into assuming that other major inventions were also responses to perceived needs. In fact, many or most inventions were developed by people driven by curiosity or by a love of tinkering, in the absence of any initial demand for the product they had in mind. Once a device had been invented, the inventor then had to find an application for it."
\begin{flushright}
Guns, Germs, and Steel: The Fates of Human Societies
\end{flushright}

\subsection{信息简史}
\begin{enumerate}
  \item 阅读194-198页,哪些人启发了香农提出熵作为信息的度量?
  \item 阅读215页,你觉得信息熵和物理熵有什么相同点?
  \item 阅读 219-225页,理解公式 $H=n\log s$ (23页第一次出现)的含义,并试计算掷三次骰子的信息熵的大小。
\end{enumerate}

参考答案:
\begin{enumerate}
  \item Harry Nyquist and John Hartley
  \item (摘录原文)Information is entropy. This was the strangest and most powerful notion of all. Entropy—already a difficult and poorly understood concept—is a measure of disorder in thermodynamics, the science of heat and energy.
  \item $3\log_2 6$
\end{enumerate}




\end{document}
