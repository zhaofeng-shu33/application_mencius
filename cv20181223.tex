\documentclass[12pt,colorlinks,linkcolor=true]{moderncv}
% set the left/right margin such that the main title can be written within one line
%% ModernCV themes
\moderncvstyle{classic}
\moderncvcolor{blue}
%\renewcommand{\familydefault}{\sfdefault}
\nopagenumbers{}
\usepackage{ctex}
\usepackage{longtable}
\usepackage{multirow}
%% Adjust the page margins
\usepackage[scale=0.75]{geometry}

%% Personal data
\firstname{\textbf{赵}}
\familyname{\textbf{丰}}
\address{深圳市南山区深圳大学城信息大楼1111A}{邮编:518000} 
\mobile{18800190762 (微信)} 
\email{616545598@qq.com} 
\homepage{https://www.douban.com/people/212256190/} 
\photo[64pt][0.4pt]{1inch.png}
\setlength{\parindent}{2em}
%%------------------------------------------------------------------------------
%% Content
%%------------------------------------------------------------------------------
\setlength\arrayrulewidth{.4pt}
\setlength\tabcolsep{6pt}
\begin{document}
\makecvtitle
\section{基本信息}
\cvitem{民族}{汉}
\cvitem{政治面貌}{共青团员}
\cvitem{籍贯}{山东枣庄}
\cvitem{出生年月}{1994/7/11}
\section{教育背景}
\cventry{2013--2017}{理学学士学位}{清华大学数学系}{北京}{}{}  % arguments 3 to 6 can be left empty
\cventry{2017至今}{}{清华大学深圳研究生院信息学部}{}{博士在读}{}
\section{读书方面兴趣爱好}
\begin{enumerate}
\item 中国小说、剧本、世界文学中的小说(中译本)、欧洲史方面(译本)、科普读物(译本)
\item 结合注释,用文言文阅读中国古代文史哲方面的著作,如诗、书、易、左传、史记等
\item 读过若干本英文小说、传记及科普读物。
\item 结合其他小语种的学习背景,有尝试阅读德语儿童文学、日语短篇小说等文本
\end{enumerate}
\section{社工社团活动相关经历}
\begin{enumerate}
\item 本科以来参与过国学社 《四书章句集注》的晨读、
经典文化传播协会 《孟子》的研读、《诗经》的背诵、以及儒士社的传统礼仪
的现代化改造等活动。
\item 本科时上过陈颖飞老师开设的《孟子》研读的课程
\item
2018年3月至2019年1月,在清华大学深圳研究生院团委先后担任紫荆志愿团副团长、联络部部长、
科协主席等职务
\item 2022年3月至今,清华大学禅学社(北京)骨干
\end{enumerate}
\section{为什么愿意作为领读人}
高三那年暑假,我拒绝了一些社交的场合和
学车等不太喜欢的活动,每天泡在家里的书房、学校的图书室或是市里的书店,
阅读各种中英文的书籍。给我印象最深的是林语堂先生写的 The Gay Genius (苏东坡传),
让我初步领略了20世纪的文豪与11世纪的文豪的对话,一方是英语、另一方是文言。
从那时此,我对英语这种现代语言与文言这种古典语言的交融怀有较大的热情,希望借
“格致计划”这个平台透过《孟子》这部著作分享我的一些心得体会。
\end{document}