\documentclass[12pt]{ctexart}
\def\version{1.0}
\newenvironment{smaller}{\zihao{5}}{\zihao{-4}}
% set the left/right margin such that the main title can be written within one line
\usepackage[left=30mm]{geometry}
\usepackage{enumitem}
\AddEnumerateCounter{\chinese}{\chinese}{}
\usepackage{fancyhdr}
\usepackage{graphicx}
\usepackage{longtable}
\fancypagestyle{runningpage}
{
  \fancyhead{}
  \fancyhead[C]{清华大学深圳研究生院紫荆志愿者团}
  \fancyfoot{}
  \fancyfoot[C]{第 \thepage 页}
}
% not works?
\ctexset {
 appendixname = {附录}
}
\def\CurriculumScheduleWidth{1.6cm}
\begin{document}
% first page is cover
\begin{titlepage}
\begin{center}
    \vspace{-0.5in}
    \textmd{\textbf{\huge{格致计划2022夏季读书营xx地点}}}\\
    \normalsize\vspace{0.1in}\Large{孟子一书及其英译本研读}\\
    \vspace{0.8in}
     \textbf{\huge{教}}\\
    \vspace{0.8in}
     \textbf{\huge{学}}\\
    \vspace{0.8in}
     \textbf{\huge{大}}\\
    \vspace{0.8in}
    \textbf{\huge{纲}}\\
    \vspace{0.8in}
    领读人:赵丰
\vspace{0.2in}
\begin{smaller}

版本:\texttt{\version}

\today
\end{smaller}
\end{center}
\end{titlepage}
\thispagestyle{empty}
\pagebreak
\pagestyle{runningpage}

\section{课程简介}
孟子这个名字绝对是家喻户晓,甚至可以说世界闻名,其人及其著作《孟子》一书
历经两千多年的传承在今天这个文化多元的世界中依然大放异彩。
可除了我们在中学语文教科书背诵过的孟子选段外,我们对
孟子其人其书还了解多少?孟子究竟有怎样的思想?作为儒家思想的重要组成部分,
他的思想对封建社会又产生了怎样深刻的影响?
在当今世界,英语世界的读者是如何透过译本了解孟子的?我们又可以从《孟子》的
英译本学到什么?如果你对这些问题抱有一定的兴趣,那不妨来听听这门课吧!
在这门课上,我们将深入到《孟子》文本之中,并对照着英文翻译逐字逐句逐段进行剖析。
通过这种方式,我们希望你的文言文和英文的阅读能力能有一定的提高,同时能够
了解孟子的思想、战国时期的社会与文化、
以及儒家体系中一些概念的英语译法。

\section{教学方法}
这门课程分为 lecture (讲课) 和 seminar (研讨)两种形式。
一共十节课,讲课部分是由我来讲,前面几节课我会先带大家阅读《梁惠王章句上下》,
而后面则主要是研讨部分,我们将把《孟子》一书剩余的文本分给每个同学,
由每个同学来讲解文本。讲解文本时是以《孟子》原文(文言文)为主,
参考朱熹的注释和解经以及英译本的全文翻译。英译本我们采用 美国汉学家范诺登
(Bryan W. Van Norden) 的翻译,这是一个21世纪的译本,比较贴合当代美国英语的使用习惯。
由于朱熹对文言字词的注释比较简略,
研读文本过程中也可以适当参考文言文字典、其他解经的古书的注释或者现代人的注释等。
考虑到学生大部分为刚参加过高考的高三同学,虽然有一定的文言文和英语基础,但可能缺少阅读
长文本的经验,在课程的前几节我会放慢研读速度,并且通过以下几点帮助同学更好地理解文本:
\begin{enumerate}
\item 尽可能多地补充一些社会历史文化方面的知识
\item 重复解释比较重要的文言实词
\item 带大家细读英译本的部分章节并分析其中一些翻译的技巧
\end{enumerate}
\section{教学目标}
本课程的重点是读文言文,
《孟子》原文需要通读一篇,
需对本书中出现的成语、典故、反映出的儒家思想有所了解。
注释和英译本不需通读,只是在遇到原文读不懂的时候参考一下。当然对于有兴趣和时间
的同学也可自行通读英译本。


\section{文本}
《四书章句集注》孟子部分, Mengzi by Van Norden (这本书带有部分朱熹注释的翻译)

\section{课程安排}
\begin{table}[!ht]
  \centering
  \begin{tabular}{|p{2cm}|p{2cm}|p{6cm}|}
  \hline
  节次 & 内容 & 备注 \\
  \hline
 第一课时 & 导论 + 梁惠王章句上导读 & 古往今来中国名人对孟子的评价;
 研读《史记·荀卿列传》中的孟子部分,了解孟子生平;
 介绍《孟子》一书的英译史;
 领读梁惠王章句上,分七部分(跳过三“寡人之于国也”),读一三四,
 针对每一部分的每句话,先读原文、再读朱熹的注释、再读原文的翻译(不读注释的翻译)
 \\
  \hline
  第二课时 & 梁惠王章句上下导读 & 针对每一部分的每句话,先读原文,选读朱熹注与原文翻译\\
  \hline
  第三课时 & 黄少平 & 越疆科技 \\
  \hline
  第四课时 & 张炎武 & 打稔村购买  \\
  \hline
  第五课时 & 赵丰、张炎武、周俞君 & 自带、购买 \\
  \hline
  第六课时 & 王鲜俐 & 自带 \\
  \hline
  第七课时 & 王鲜俐 & 自带 \\
  \hline
  第八课时 & 王鲜俐 & 自带 \\
  \hline
  第九课时 & 王鲜俐 & 自带 \\
  \hline
  第十课时 & 总结孟子思想 & 自带 \\
  \hline
\end{tabular}
  \caption{课程安排}
\end{table}

\end{document}
